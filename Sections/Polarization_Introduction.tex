\label{sec:INTRO}
Most sources of light can be characterized as non polarized, de-coherent and transversal electromagnetic waves. This means that light rays - for instance from the sun - are composed of transversal electromagnetic oscillations with random spatial orientations and temporal decoherence. However, in its simplest and purest form, light has a preferred spatial orientation as shown in figure \ref{fig:PolvNPol}. This state of light is referred to as polarized light.\\

This work aims to explain a complete process of characterizing the polarization of a LASER beam. This is with the objective of reinforcing the basic principles behind polarization through hands-on experimentation. The team creates a empirical argument of how polarized light behaves when subjected to different optical elements. 